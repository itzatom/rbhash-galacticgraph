\def\baselinestretch{1}
\section{Formato di input e di output}
\def\baselinestretch{1.66}
\thispagestyle{headings}

\subsection{Input}
I dati in input del problema sono:
\begin{itemize}
    \item V: numero intero di Vertici del grafo
    \item E: numero intero di Archi del grafo
    \item W: numero intero di Wormholes presenti nel grafo
    \item Tuple rappresentante archi: NodoA, NodoB, Peso
    \footnote{ndr: NodoA e NodoB sono le chiavi intere dei vertici
    e Peso \`e un intero usato per rappresentare il peso di tale
    arco.}
\end{itemize}
Tali dati sono immagazzinati in un file di testo non binario
contenente nel primo rigo i primi tre dati elencati, mentre nei
successivi sono presenti le \textbf{tuple}.
Per rappresentare i wormhole il programma prende gli \textbf{
ultimi W NodiB} contenuti nel file e li va a salvare in un
vettore di vertici.

\subsection{Output}
Il programma sviluppato restituisce in output i nodi che collegano
la coppia \textbf{sorgente - destinazione} nel minor "tempo" possibile 
e il relativo costo di tale cammino minino, \textbf{se esiste}:
pu\`o capitare, come vedremo nel paragrafo \textit{"Test effettuati"},
che il grafo non sia connesso e che il nodo destinazione sia raggiungibile 
solo attraverso i vertici di tipo wormhole.
Inoltre il programma restituisce, il cammino minimo (vertici da attraversare
e costo totale) facendo uso dei nodi speciali wormhole. L'output secondario
pu\`o mancare nel caso in cui non si incontrino wormhole, oppure il wormhole di partenza \`e uguale a quello di destinazione.