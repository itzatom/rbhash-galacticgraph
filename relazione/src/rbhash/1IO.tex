\def\baselinestretch{1}
\section{Formato di input e di output}
\def\baselinestretch{1.66}
\thispagestyle{headings}

\subsection{Input}
\indent Il programma lavora su tuple nel formato $chiave1:chiave2:stringa$ e i dati in input sono dati da:
\begin{itemize}
    \item file di testo da selezionare all'avvio contenente le tuple;
    \begin{minted}{bash}
        $ ./rbhash inputs/asd.txt
    \end{minted}
    \item opzione da tastiera per effettuare operazioni sulle tuple.
    \begin{minted}{bash}
    >_ 1:2:stringa
    \end{minted}
\end{itemize}
Il programma legger\`a le righe e allocher\`a nodi e hashtable in base alle chiavi date in input.
\subsection{Output}
L'output del programma \`e un menu contestuale in cui l'utente pu\`o effettuare le operazioni
mostrate di seguito:


\begin{minted}[bash]
  **** MENU ****

   1. Insert
   2. Remove
   3. Query
   4. Print
   0. Exit
  >_
\end{minted}


\noindent Tutte le opzioni restituiscono un output di avvenuta operazione con dettagli su eventuali errori,
ad eccezzione della stampa (opzione 4) che restituisce in output l'intera struttura dati caricata
in memoria.