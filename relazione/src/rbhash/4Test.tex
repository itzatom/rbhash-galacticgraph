\def\baselinestretch{1}
\section{Test e risultati}
\def\baselinestretch{1.66}
\thispagestyle{headings}

\subsection{Test effettuati}
Nei test effettuati sono stai usati due file e tramite funzioni di debug per 
la stampa si \`e cercato di verificare per quanto possibile l'inserimento
e cancellazione dell'albero red black. Di seguito verranno mostrati 
alcuni esempi estrapolati dalla shell.

\begin{minted}[bash]
123:321:hello
2:4:54
1:4:test
2:4:h&:o
2:12:asd
\end{minted}


Notare come nell'output di stampa della struttura dati non vi sono presenti le chiavi
duplicate. Inoltre si \`e scelto di stampare tramite una \textbf{visita inorder} per avere un ordine crescente delle
chiavi dell'albero.\newline
\begin{minted}{bash}
$ ./rbhash inputs/asd

**** MENU ****
 1. Insert
 2. Remove
 3. Query
 4. Print
 0. Exit
>_ 4

 '(1) :
        [4] : test'

 (2) :
        [4] : 54
        [12] : asd

'(123) :
        [321] : hello'
        
\end{minted}

L'eliminazione della radice, ovvero il nodo nero $2$ comporta un bilanciamento con
conseguente ricolorazione di nodi.


\begin{minted}{bash}
*****WARNING*****
Insert tuple (key1:key2:data) to delete -> 2:4:54

*****WARNING*****
Insert tuple (key1:key2:data) to delete -> 2:12:asd


'(1) :
        [4] : test'
(123) :
        [321] : hello

\end{minted}