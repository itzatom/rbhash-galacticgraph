\def\baselinestretch{1}
\chapter{Albero red-black di hash table}
\def\baselinestretch{1.66}
\thispagestyle{headings}

\def\baselinestretch{1}
\section{Descrizione problema}
\def\baselinestretch{1.66}
\thispagestyle{headings}

Il problema in analisi prevede di creare una struttura dati, in C++, che d'ora in avanti chiameremo
\textbf{red-black hash}, in grado di immagazzinare delle stringhe alfanumeriche. Tale struttura
\`e l'unione di un albero binario di ricerca bilanciato, \textbf{albero rosso-nero} o \textbf{albero
red-black}, e delle \textbf{hash table}: all'interno di ogni nodo di tale albero,
vi \`e presente una hash table, struttura dati che associa per ogni chiave un singolo valore, al cui
interno sono presenti delle stringhe. La traccia prevedeva di poter effettuare operazioni
\textbf{C.R.D.}\footnote{Create Retrieve Delete. Operazioni tipiche delle basi di dati, ma senza la
possibilit\`a di effettuare Updates.} su tuple nel formato: \verb|chiave1:chiave2:stringa| . 
La chiave 1 indicizza un nodo dell'albero red black, il quale puntando ad una hash
table utilizza la chiave 2 per associare la stringa.
Vi \`e quindi una relazione \verb|1:1| per i nodi dell'albero e l'hash table, e \verb|1:M| tra l'hash table
e le stringhe, dove \textbf{M} \`e la dimensione massima dell'hash table.

\def\baselinestretch{1}
\section{Descrizione strutture dati}
\def\baselinestretch{1.66}
\thispagestyle{headings}
\subsection{Alberi binari di ricerca}
\indent Gli \textbf{alberi binari di ricerca} sono delle strutture dati
 che immagazzinano dati in un albero avente in ogni nodo due 
 figli. Gli \textbf{ABR} (o in inglese BST) godono della seguente propriet\`a:
 $\forall \ x \in \ $BST$ :\ key(x.left) \leq key(x) < key(x.right)$. 
 Ovvero ogni nodo in un ABR ha come valore della chiave un valore
 maggiore del figlio sinistro ma minore di quello destro.
 \newline Ci\`o assicura operazioni in una complessit\`a 
 \textbf{logarimtica} data dalla profondit\`a dell'albero.
 Il problema sorge nel caso in cui avvengono cancellazioni
 sbagliate o inserimenti sbagliati che seppur mantengono
 la propriet\`a degli ABR, degradano tale albero in una lista concatenata
 compromettendo le operazioni di inserimento, cancellazione
 e ricerca ad avere complessit\`a lineare.

\subsection{Alberi Red-Black}
Gli alberi red black sono degli alberi binari di ricerca
\textbf{ autobilancianti }. Ogni volta che si inserisce un nuovo 
nodo, o lo 
si cancella si effettuano delle operazioni per bilanciare 
l'albero. Gli alberi rosso neri posseggono \textbf{5 propriet\`a} 
utili
ai metodi di supporto $insertFix()$ e $deleteFix()$ per garantire che le complessit\`a
peggiori abbiano al pi\`u come valore l'altezza dell'albero, ovvero $O(log_2n)$. 
I metodi di supporto all'inserimento e alla cancellazione fanno uso delle rotazioni di un nodo,
operazione che permette di compattare l'albero e garantire il corretto bilanciamento.
Di seguito verranno elencate tali propriet\`a:
\begin{itemize}
    \item ogni nodo \`e rosso o nero;
    \item la radice \`e nera;
    \item ogni foglia \`e nera;
    \item se un nodo \`e rosso, allora entrambi i suoi figli sono neri;
    \item per ogni nodo, tutti i cammini semplici che vanno dal nodo alle foglie sue discendenti
    contengono lo stesso numero di nodi neri.
\end{itemize}

\subsection{Hash Table ad indirizzamento aperto}
Le \textbf{Hash Table}, o hash map, sono delle strutture dati che permettono di associare ad una
chiave un singolo valore. Precisamente ad ogni chiave va applicata una funzione detta funzione
di hashing che calcoler\`a un indice. Pu\`o capitare per\`o che una funzione hash applicata su chiavi diverse 
indicizzi celle simili dell'hashtable, per tanto bisogna gestire queste \textit{collisioni}.
La tavola hash realizzata \`e del tipo a \textbf{indirizzamento aperto}, ovvero non facendo
uso dei puntatori si \textit{ispeziona} l'hashtable fino a incontrare una posizione libera,
se presente. Il metodo di ispezione scelto \`e quello del \textbf{doppio hashing}: rispetto
ad altre ispezioni, quella del doppio hashing trova una posizione in modo pi\`u veloce.
Nei capitoli successivi vedremo nel dettaglio il funzionamento.
