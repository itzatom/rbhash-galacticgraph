\def\baselinestretch{1}
\section{Studio complessit\`a}
\def\baselinestretch{1.66}
\thispagestyle{headings}
La complessit\`a di tempo data dall'algoritmo galactic dijkstra \`e influenzata
dalle \textbf{due esecuzioni} (al pi\`u) del classico algoritmo di Dijkstra. Il calcolo
dei percorsi impiega $O(3E)$ poich\'e per raggiungere un nodo fino alla sorgente vuol dire
ripercorrere l'albero attraversando ogni arco nel caso peggiore, quindi mettendoci un tempo lineare.
Per cui si considerano le due applicazioni di Dijkstra che impiegano $O(2(V+E) 2log_2 V)$ che diventa
$O(2E2log_2 V)$ se ogni vertice \`e raggiungibile dalla sorgente, ma poich\`e è possibile portare
le costanti moltiplicative fuori diremo che la complessit\`a finale sar\`a $\mathbf{O(Elog_2V)}$.
L'algoritmo di Dijkstra impiega tale complessit\`a poich\'e influenzato dalle operazioni
$extractMin()$ e $decreaseKey()$ impiegate $|E|$ volte dalla coda di min priorit\`a basata
su min heap.
Si poteva pensare di usare un Heap di Fibonacci poich\'e si \`e notato che l'algoritmo
effettua pi\`u operazioni di $decreaseKey()$ che di $extractMin()$, e nella struttura dati
citata tale operazioni hanno rispettivamente costo computazionale $O(1)$ e $O(log_2V)$. Ci\`o
avrebbe comportato un miglioramento nel caso in cui ci fossero stati molti archi, avendo un costo
asintotico pari a $O(Vlog_2V +E)$.